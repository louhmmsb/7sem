% Created 2022-03-20 Sun 01:14
% Intended LaTeX compiler: pdflatex
\documentclass[11pt]{article}
\usepackage[utf8]{inputenc}
\usepackage[T1]{fontenc}
\usepackage{graphicx}
\usepackage{longtable}
\usepackage{wrapfig}
\usepackage{rotating}
\usepackage[normalem]{ulem}
\usepackage{amsmath}
\usepackage{amssymb}
\usepackage{capt-of}
\usepackage{hyperref}
\author{Lourenço Bogo}
\date{\today}
\title{Introdução aos Processos Estocásticos}
\hypersetup{
 pdfauthor={Lourenço Bogo},
 pdftitle={Introdução aos Processos Estocásticos},
 pdfkeywords={},
 pdfsubject={},
 pdfcreator={Emacs 29.0.50 (Org mode 9.5.2)}, 
 pdflang={English}}
\begin{document}

\maketitle

\section{Motivação}
\label{sec:org6c1afc7}
Considere que há um fenômeno se desenvolvendo aleatoriamente no decorrer do tempo.
Então, a sequência ordenada \((X_1, X_2, \dots, X_n)\) representa a evolução desse fenômeno, ou sistema, nos instantes \(1, 2, \dots, n\).
Essas variáveis não são necessariamente iid.

Temos, portanto, o conceito de um processo e não apenas evento. Daí o nome processo aleatório ou processo estocástico.

\begin{itemize}
\item Quando n cresce, qual é o comportamento da sequência?
\end{itemize}

Às vezes, o índice pode representar, não o tempo, mas uma localização no espaço \(\rightarrow\) campo aleatório.

Em outras situações pode-se ter um processso duplamente indexado, um índice representando o espaço e o outro, o tempo, ou seja, temos um processo estocástico espacial.

\section{Definição}
\label{sec:orgdbad375}
\section{Cadeia de Markov}
\label{sec:orge80c14c}
Um processo estocástico, com espaço de estados discreto S eem tempo discreto, é uma cadeia de Markov se possui a propriedade Markoviana, isto é, para todo i0, i1, \ldots{}m in-1, i \(\in\) S,

\(P(X_{n+1} = j | X_0 = i_0, X_1 = i_1 ...) = P(X_{n+1} = j | X_n = i) = p_{ij}(n)\)

Basicamente o passado é irrelevante, o próximo estado depende apenas do atual.

Uma cadeia é dita set homogênea no tempo (ou estacionaŕia) se, para \(\forall n \geq 0\),

\(p_{ij}(n) = P(X_{n+1} = j | X_n = i) = P(X_1 = j | X_0 = i) = p_{ij}(0) = p_{ij}\).

Ou seja, ir de estado \(i\) para \(j\) tem sempre a mesma probabilidade independente do tempo no qual essa transição acontece.

Essas probabilidades de transição podem ser representadas por uma matriz \(P\), chamada matrizde probabilidade de transição, cujos elementos \(p_{ij}\) satisfazem:

\begin{itemize}
\item \(p_{ij} > 0\) para todo \(i, j \in S\)
\item sum p\textsubscript{ij} sobre \(j \in S = 1\) para todo \(i \in S\), ou seja, a soma dos elements de cada linha é 1 e isso vale para toda linha.
\end{itemize}
\end{document}